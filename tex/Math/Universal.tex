\section{快速幂}
\subsection{递归形式}
\begin{lstlisting}
#define mod 1000000007
long long quick_pow(long long a, long long b)
{
    if (b == 0)
        return 1;
    long long temp = quick_pow(a, b >> 1);
    if (b & 1)
        return a % mod * temp % mod * temp % mod;
    else
        return temp % mod * temp % mod; //快速幂
}
\end{lstlisting}

\subsection{循环形式}
\textbf{如果递归形式栈溢出 可使用循环形式}
\begin{lstlisting}
#define mod 1000000007
long long quick_pow(long long x,long long n)
{
    long long ret=1;
    long long temp=x%mod;
    while(n)
    {
        if(n&1)
        {
            ret=(ret*temp)%mod;
        }
        temp=(temp*temp)%mod; //偶次
        n>>=1;
    }
    return ret;//结果
}
\end{lstlisting}

\section{矩阵快速幂}
\begin{lstlisting}
#define N 105
int m;//矩阵阶
struct matrix
{
    long long a[N][N];
    matrix(){memset(a, 0, sizeof(a));}
};
matrix matrix_mul(matrix m1, matrix m2)
{
    matrix ans;
    for (int k = 0; k < m; k++)
    {
        for (int i = 0; i < m; i++)
        {
            if (m1.a[i][k]) //剪枝
            {
                for (int j = 0; j < m; j++)
                {
                    if (m2.a[k][j]) //剪枝
                    {
                        ans.a[i][j] = (ans.a[i][j] + (m1.a[i][k] * m2.a[k][j]) % mod) % mod;
                    }
                }
            }
        }
    }
    return ans;
}
matrix quick_pow_matrix(matrix m1, long long k)
{
    matrix ans;//递归写法 可能堆栈溢出
    for (int i = 0; i < m; i++)
        ans.a[i][i] = 1;
    while (k)
    {
        if (k & 1)
            ans = matrix_mul(ans, m1);
        m1 = matrix_mul(m1, m1);
        k >>= 1;
    }
    return ans;
}
\end{lstlisting}

\section{组合数取模}
\begin{lstlisting}
#define N 100010
#define mod 1000000007
#define ll long long
ll fac[N];//阶乘
ll inv[N];//阶乘逆元
void init()
{
    fac[0]=1;
    for(int i=1;i<N;i++)
    {
        fac[i]=(fac[i-1]*i)%mod;
    }
    inv[N-1]=quick_pow(fac[N-1],mod-2);//费马小定理 求逆元a^(p-1)%p=1 a的逆元为a^(p-2)
    //如果 ax%p=1 , 那么x的最小正整数解就是 a 模p的逆元
    for(int i=N-2;i>=0;i--)
    {
        inv[i]=inv[i+1]*(i+1)%mod;
    }
}
ll C(ll a,ll b)
{
    if(b>a) return 0;
    if(b==0) return 1;
    return fac[a]*inv[b]%mod*inv[a-b]%mod;
}
\end{lstlisting}