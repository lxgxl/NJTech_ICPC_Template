\section{Tarjan算法与有向图连通性}
\subsection{强连通分量(SCC)判定法则}
\begin{lstlisting}
void Tarjan(int x)
{
    dfn[x]=low[x]=++num;
    stack[++top]=x,in_stack[x]=true;
    for(int i=head[x];i;i=nxt[i])
    {
        int y=ver[i];
        if(!dfn[y])
        {
            Tarjan(y);
            low[x]=min(low[x],low[y]);
        }
        else if(in_stack[y])
            low[x]=min(low[x],dfn[y]);
    }
    if(dfn[x]==low[x])
    {
        cnt++;
        int y;
        do
        {
            y=stack[top--],in_stack[y]=false;
            color[y]=cnt, scc[cnt].push_back(y);
        } while (x!=y);
    }
}
\end{lstlisting}
\subsection{SCC -> DAG}
\begin{lstlisting}
void SCC()
{
    for (int i = 0; i <= n; i++)
        if (!dfn[i])
            Tarjan(i);
    //缩点
    for (int x = 1; x <= n; x++)
    {
        for (int i = head[x]; i; i = nxt[i])
        {
            int y = ver1[i];
            if (color[x] != color[y])
                add_c(color[x], color[y]);
        }
    }
}
\end{lstlisting}
例题分析

POJ1236 Network of Schools(SCC->DAG,入度出度)

P3275 [SCOI2011]糖果(SPFA TLE,SCC->DAG,Topo,DP)

\subsection{有向图的必经点与必经边}