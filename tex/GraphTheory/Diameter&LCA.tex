\section{树的直径}
\subsection{树形DP求树的直径}
仅能求出直径长度,无法得知路径信息,可处理负权边。
\begin{lstlisting}
int dp[N];
//dp[rt] 以rt为根的子树 从rt出发最远可达距离
/*
    对于每个结点x f[x]:经过节点x的最长链长度
*/
void DP(int rt)
{
    dp[rt]=0;//单点
    vis[rt]=1;
    for(int i=head[rt];i;i=nxt[i])
    {
        int s=ver[i];
        if(!vis[s])
        {
            DP(s);
            diameter=max(diameter,dp[rt]+dp[s]+edge[i]);
            dp[rt]=max(dp[rt],dp[s]+edge[i]);
        }
    }
}
\end{lstlisting}
\subsection{两次BFS/DFS求树的直径}
无法处理负权边,容易记录路径
\begin{lstlisting}
void DFS(int start,bool record_path)
{
    vis[start]=1;
    for(int i=head[start];i;i=nxt[i])
    {
        int s=ver[i];
        if(!vis[s])
        {
            dis[s]=dis[start]+edge[i];
            if(record_path) path[s]=i;
            DFS(s,record_path);
        }
    }
    vis[start]=0;//清理
}
\end{lstlisting}
例题分析

P3629 [APIO2010]巡逻(两种求树直径方法的综合应用)

P1099 树网的核(枚举)

\section{最近公共祖先(LCA)}
\subsection{树上倍增}
\begin{lstlisting}
void BFS()
{
    queue<int> q;
    q.push(1);
    d[1] = 1;
    while (!q.empty())
    {
        int x = q.front();
        q.pop();
        for (int i = head[x]; i; i = nxt[i])
        {
            int y = ver[i];
            if (!d[y])
            {
                d[y] = d[x] + 1;
                fa[y][0] = x;
                for (int j = 1; j <= k; j++)
                {
                    fa[y][j] = fa[fa[y][j - 1]][j - 1];
                }
                q.push(y);
            }
        }
    }
}
int LCA(int x, int y)
{
    if (d[x] < d[y])
        swap(x, y);
    for (int i = k; i >= 0; i--)
        if (d[fa[x][i]] >= d[y])
            x = fa[x][i];
    if (x == y)
        return y;
    for (int i = k; i >= 0; i--)
        if (fa[x][i] != fa[y][i])
            x = fa[x][i], y = fa[y][i];
    return fa[x][0];
}
\end{lstlisting}
\subsection{Tarjan算法}
\begin{lstlisting}
int Find(int x)
{
    if (x == fa[x])
        return x;
    return fa[x] = Find(fa[x]);
}
void Tarjan(int x)
{
    vis[x] = 1;
    for (int i = head[x]; i; i = nxt[i])
    {
        int y = ver[i];
        if (!vis[y])
        {
            Tarjan(y);
            fa[y] = x;
        }
    }
    for (int i = 0; i < q[x].size(); i++)
    {
        int y = q[x][i].first, id = q[x][i].second;
        if (vis[y] == 2)
            lca[id] = Find(y);
    }
    vis[x] = 2;
}
\end{lstlisting}

\section{树上差分与LCA的综合应用}
